\begin{filecontents*}{\jobname.bib}
@article{Worsch_2009_AUC_ar,
  author  = {Thomas Worsch and Hidenosuke Nishio},
  title   = {Achieving universality of {CA} by changing the neighborhood},
  journal = {Journal of Cellular Automata},
  year    = {2009},
  volume  = {4},
  number  = {3},
  pages   = {237--246},
}
@inproceedings{Worsch_2012_IUA_ip_acri,
  author    = {Thomas Worsch},
  title     = {({I}ntrinsically?) Universal Asynchronous Cellular Automata},
  editor    = {Georgios Sirakoulis and Stefania Bandini},
  booktitle = {Proceedings ACRI 2012},
  year      = {2012},
  pages     = {689--698},
  publisher = {Springer},
  series    = {LNCS},
  volume    = {7495},
}
\end{filecontents*}

\documentclass[11pt]{article}

\usepackage[T1]{fontenc}
\usepackage[ngerman]{babel}
\usepackage[utf8]{inputenc}

\usepackage{amsmath}
% \usepackage{mathtools}

\usepackage[tt=false]{libertine}   % !!!!! das muss man nicht nutzen
\usepackage[libertine]{newtxmath}  % !!!!! das muss man nicht nutzen
%\usepackage[supstfm=libertinesups,supscaled=1.2,raised=-.13em]{superiors} % params taken from doc

%
%\usepackage{tgpagella}
%\usepackage[euler-digits]{eulervm}
%
\usepackage{microtype}

\usepackage{fancyvrb}

%\usepackage{graphicx}

\usepackage{booktabs}
\usepackage[shortlabels]{enumitem}
\setlist{noitemsep}

\usepackage{titlesec}
\usepackage{tcolorbox}
\tcbuselibrary{listingsutf8}

\usepackage{bbold}
\newcommand{\Z}{\mathbb{Z}}


%-----------------------------------------------------------------------------
% für das Deckblatt

\usepackage{tikz}

\newcommand{\teilnehmername}{Vorname Nachname} % !!!!!
\newcommand{\teilnehmermatrnr}{1234567}        % !!!!!
\newcommand{\seminarart}{Proseminar}           % !!!!!  oder Seminar
\newcommand{\seminarlp}{3 LP}                  % !!!!!  Prosem immer 3 LP, Sem 3 oder 4 LP
\newcommand{\seminarjahr}{2017}                % !!!!!
%-----------------------------------------------------------------------------
\newcommand{\meta}[1]{$\langle$\textit{#1}$\rangle$}
\newcommand{\paket}[1]{\texttt{#1}}
\newcommand{\prgname}[1]{\texttt{#1}}

%-----------------------------------------------------------------------------
\author{Thomas Worsch}
\title{Hinweise zu Seminarausarbeitungen}

%=============================================================================
\begin{document}
%=======================================================================
% Anfang erste Seite
{\thispagestyle{empty}\large\sffamily\raggedright
%
\begin{tikzpicture}[remember picture,overlay]
  \coordinate[xshift=5mm,yshift=-5mm] (NW) at (current page.north west) {};
  \coordinate[xshift=-5mm,yshift=-5mm] (NE) at (current page.north east) {};
  \coordinate[xshift=-5mm,yshift=13mm] (SE) at (current page.south east) {};
  \coordinate[xshift=5mm,yshift=13mm] (SW) at (current page.south west) {};

  \draw[line width=0.25pt] (NW)
    [rounded corners=5mm] -- (NE) 
    [sharp corners] -- (SE)
    [rounded corners=5mm] -- (SW)
    [sharp corners] -- cycle
  ;
\end{tikzpicture}
%
\unskip % keine Ahnung warum das nötig ist
\noindent \textbf{\Large \seminarart\ (\seminarlp)} 
\\[\baselineskip]
%
Zellularautomaten und diskrete komplexe Systeme
% für Fortgeschrittene  % nur für das 4 Leistungspunkte Seminar !!!!!
\\[1ex]
%
im Sommersemester \seminarjahr

\vspace*{3\baselineskip}

\noindent \textbf{\Large Ausarbeitung} \\[\baselineskip]
%
von \textbf{\teilnehmername}, Matr.nr.~\teilnehmermatrnr

\vspace*{3\baselineskip}

\noindent \textbf{\Large Thema} \\[\baselineskip]
%
% nachfolgende ein Beispiel, für Konferenzbeiträge, Buchausschnitte, ...
% bitte analog vorgehen !!!!!
%
Karl Gustav von Kloppenstöck (1958)\\[1ex]
%
\textit{Zellularautomaten im ausgehenden Kambrium unter besonderer
  Berücksichtigung der Karlsruher Unterpflasterstraßenbahnen}\\[1ex]
%
Zentralblatt für historischen ÖPNV, Band \textbf{42}, S.~123-456
}
\clearpage
% Ende erste Seite
%=======================================================================
% Anfang zweite Seite
{\thispagestyle{empty}\raggedright

\noindent \textbf{\Large Erklärung}\\[1ex]
gemäß \S 6 (11) der Prüfungsordnung Informatik % !!!!! oder \S 6 (7) (bei MasterPO 2015)
(Bachelor) % oder Master !!!!!
\\[\baselineskip]

\noindent
Ich versichere wahrheitsgemäß, die Seminarausarbeitung zum
\seminarart{} "`Zellularautomaten und diskrete komplexe Systeme"' im
Sommersemester \seminarjahr{} selbstständig angefertigt, alle
benutzten Hilfsmittel vollständig und genau angegeben und alles
kenntlich gemacht zu haben, was aus Arbeiten anderer unverändert oder
mit Abänderungen entnommen wurde.

\vspace*{30mm}
\noindent
\begin{tabular}{@{}l}
  \hline
   \\[-1ex]
  \hbox to 0.6\textwidth{(\teilnehmername, Matr.nr.~\teilnehmermatrnr) \hss}
\end{tabular}
}
\clearpage
% Ende zweite Seite
%=======================================================================

%-----------------------------------------------------------------------------
\section{Einleitung}

Hoffentlich geht das gut: Das vorliegende Dokument soll gleichzeitig mehreren
Zwecken dienen:
%
\begin{itemize}[noitemsep]
\item Es soll als Vorlage für Seminarausarbeitungen dienen.
  %
  Daher stelle ich Ihnen die \LaTeX"=Quelle zur Verfügung.
  %
  Deshalb sind die beiden ersten Seiten auch etwas ungewöhnlich.
\item Es soll ein paar (meiner Meinung nach) wichtige Dinge zum Thema \LaTeX{}
  vermitteln.
\item Es soll dafür sorgen, dass der in den Seminarausarbeitungen benutzte
  Formalismus (halbwegs) einheitlich ist.
\end{itemize}
%
Daher ist die weitere Arbeit ist wie folgt aufgebaut: 

In Abschnitt~\ref{sec:grundlagen} sind noch mal die Grundlagen zu
Zellularautomaten aufgeschrieben. Die Auf"|forderung an Seminarteilnehmer ist:
%
\begin{itemize}[noitemsep]
\item Bitte benutzen Sie diesen Formalismus (soweit möglich und sinnvoll).
\item Aber bitte schreiben Sie ihn nicht alle noch mal ab, sondern verweisen
  Sie (erst mal) auf das vorliegende Dokument.
\item Allerdings ist auch mir noch unklar: Wie heißt dieses Dokument? Wie soll
  man das ziteren?
\end{itemize}
%
In Abschnitt~\ref{sec:dokument-struktur} wird erläutert, an welchen Stellen
man in diesem Dokument Dinge findet, die für den gewählten Aufbau von
(Pro-)Seminarausarbeitungen von Bedeutung sind.

In Abschnitt~\ref{sec:benutzte-pakete} findet man einige Anmerkungen zu den in
diesem \LaTeX"=Dokument benutzten Paketen. Da lernen Sie vielleicht noch das
ein oder andere für Ihr späteres Leben.

In Abschnitt~\ref{sec:no-go} werden einige böse Dinge aufgezählt, die man
\emph{niemals} tun soll.

In Abschnitt~\ref{sec:tipps} finden sich ein paar allgemeine Tipps zur
Abfassung von \LaTeX"=Quellen.

In der aktuellen Version dieses Dokumentes gibt es noch Lücken. Die, die ich
sehe, sind abschließend in Abschnitt~\ref{sec:todo} aufgeführt.

(Randbemerkung: Es ist eine gute und übliche Vorgehensweise, den ersten
einleitenden Abschnitt einer Arbeit wie eben geschehen mit einem Überblick
über den Rest zu beenden.)

%-----------------------------------------------------------------------------
\section{Grundlagen}
\label{sec:grundlagen}

$\Z$ bezeichnet die Menge der ganzen Zahlen. Sind $A$ und $B$ zwei Mengen, so
schreiben wir $B^A$ für die Menge aller Abbildungen der Form $f \colon A\to
B$.

Ein Zellularautomat ist festgelegt durch
%
\begin{enumerate}[noitemsep]
\item den zugrunde liegenden Raum $R$,
\item die endliche Zustandsmenge $Q$,
\item die endliche Nachbarschaft $N$ und
\item die lokale Überführungsfunktion
  \[
  \delta: Q^N \to Q
  \]
\end{enumerate}
%
Erläuterungen und abgeleitete Begriffe:
%
\begin{itemize}[noitemsep]
\item $R$ ist bei uns meist $\Z$ oder $\Z_m$ oder $\Z^2$
\item Formal enthält $N$ Koordinaten\emph{differenzen}.
\item Eine \emph{lokale Konfiguration} ist eine Abbildung $\ell:N\to Q$, also
  mit anderen Worten $\ell\in Q^N$.
\item Die lokale Überführungsfunktion induziert eine globale
  Überführungsfunktion
  \[
  \Delta: Q^R \to Q^R
  \]
\end{itemize}
%
Bitte malen Sie Raum"=Zeit"=Diagramme immer so, dass die Zeit \emph{von oben
  nach unten} zunimmt.


%-----------------------------------------------------------------------------
\section{Anmerkungen zur Struktur der \LaTeX"=Quelle dieses Dokuments}
\label{sec:dokument-struktur}

\subsection{Allgemeines}

Jedes \LaTeX"=Dokument hat folgende sytaktische Grobstruktur:

\begin{tcolorbox}
\begin{Verbatim}[commandchars=\@\[\]]
\documentclass{article}
  @meta[Präambel]
\begin{document}
  @meta[Rumpf]
\end{document}
\end{Verbatim}
\end{tcolorbox}

Vor der Festlegung \verb|\documentclass{|\dots\verb|}| sind nur wenige
Kommandos erlaubt.
%
Eine Möglichkeit ist die Benutzung einer sogenannten \emph{Umgebung} namens
\verb|filecontents*|.

\begin{tcolorbox}
\begin{Verbatim}[commandchars=\@\[\]]
\begin{filecontents*}{@meta[Dateiname]}
  @meta[Dateiinhalt]
\end{filecontents*}
\end{Verbatim}
\end{tcolorbox}

Im vorliegenden Fall benutzen wir sie, um in die \LaTeX"=Quelle auch gleich
noch eine Datei mit Angaben zu Literaturquellen mit einzubinden.
%
Die erzeugte Datei hat den gleichen Namen wie die \LaTeX"=Quelle, aber mit der
Endung \verb|.bib|.


\subsection{Die beiden ersten Seiten für unsere Ausarbeitungen}


Bitte beachten Sie, dass Sie sowohl in der Präambel als auch auf der ersten
Seite in den mit fünf Ausrufezeichen \texttt{!!!!!} gekennzeichneten Zeilen
auf jeden Fall Anpassungen vornehmen müssen.


%-----------------------------------------------------------------------------
\section{Anmerkungen zu benutzten Paketen}
\label{sec:benutzte-pakete}

Es gibt verschiedene sogenannte \emph{"`\TeX{} engines"'}.
%
Der derzeitige Aufbau dieses Dokumentes, genauer gesagt die benutzte Auswahl
von \LaTeX"=Paketen geht davon aus, dass \prgname{pdflatex} benutzt wird.
%
Erfahrene Studenten, die \prgname{lualatex} oder \prgname{xelatex} nutzen
(wollen), werden vermutlich wissen, dass dann die Paketauswahl anders scein
muss.

%.............................................................................
\subsection{Paket \paket{fontenc}}

Am besten einfach wie in der Präambel dieses Dokuments angegeben benutzen:

\verb|\usepackage[T1]{fontenc}|

\subsection{Paket \paket{inputenc}}

In der Präambel dieses Dokumentes steht:

\verb|\usepackage[utf8]{inputenc}|

Das ist richtig so, weil die \LaTeX"=Quelle UTF-8-codiert abgespeichert
ist. Wenn man die Codierung "'ISO latin 1"' benutzt, dann muss es in der
Präambel heißen:

\verb|\usepackage[latin1]{inputenc}|

%.............................................................................
\subsection{Paket \paket{babel}}

\paragraph{Trennungen.}

Die Option \verb|ngerman| sorgt dafür, dass sich der automatische
Trennalgorithmus an die deutschen "`Regeln"' hält, \dots\ jedenfalls weitgehend.

Wird ein Wort trotzdem falsch getrennt, in der Präambel eine Zeile der
folgenden Form einfügen:
\begin{itemize}
\item \verb|\hyphenation{Tu-ring-ma-schi-ne}|
\end{itemize}

\paragraph{Anführungszeichen.}
%
\begin{itemize}[noitemsep]
\item normale deutsche gehen so: am Anfang \verb|"`| und am Ende \verb|"'|
\item Beispiel: \verb|"`Hallo!"'| liefert "`Hallo!"'
\item eine Alternative sind am Anfang \verb|"<| und am Ende \verb|">|
\item Beispiel: \verb|"<Hallo!">| liefert "<Hallo!">
\end{itemize}


\paragraph{Bindestriche.}

Wenn man in einem Wort einen Bindestrich als "`Minuszeichen"' \verb|-|
eingibt, dann sind die entsprechenden Stellen \emph{die einzigen} Stellen, an
denen \TeX{} noch trennt.

Wenn man den Bindestrich in der Form \verb|"=| notiert, man bleiben die
Trennstellen in den Wortteilen erhalten. Zum Vergleich nehmen wir das Wort
\emph{Turingmaschinen"=Konstruktor}:
\begin{itemize}
\item \verb|Turingmaschinen"=Konstruktor| liefert\\
  gaaaaaaaaaaaaaaaaaaaaaaaaaaaaaaaaaaaaaaaaaaaaanz hinten Turingmaschinen"=Konstruktor
  aber
\item  \verb|Turingmaschinen-Konstruktor| liefert\\
  gaaaaaaaaaaaaaaaaaaaaaaaaaaaaaaaaaaaaaaaaaaaaanz hinten Turingmaschinen-Konstruktor
\item
  Hier findet \TeX also die hässliche überhängende Zeile immer noch "`hübscher"'
  als die zu kurze:\\
  gaaaaaaaaaaaaaaaaaaaaaaaaaaaaaaaaaaaaaaaaaaaaanz hinten \\
  Turingmaschinen-Konstruktor
\item Randbemerkung: das "`Wort"' mit den vielen aaaaaaa enthält in allen drei
  Fällen gleich viele! (Die beobachtbare Stauchung verdankt man
  \paket{microtype}; siehe weiter unten.)
\end{itemize}

\paragraph{Ligaturen.}

\LaTeX{} macht automatisch Ligaturen. Das bedeutet, dass manchmal zwei
Buchstaben zu einem Zeichen zusammengezogen werden. Hier klassische Beispiele
in ganz groß:
%
\begin{center}
  \scalebox{3}{fl ff ffl}
\end{center}
%
Deutsche Typografie will das aber \emph{nicht}, wenn zum Beispiel das \verb|f|
und das \verb|l| zu verschiedenen Wortteilen gehören.  Man möchte 
%
\begin{center}
  \scalebox{3}{Auf"|lage, \emph{nicht} Auflage}
\end{center}
%
Um unerwünschte Ligaturen zu verhindern, benutzt man \verb="|= zwischen den
betroffenen Buchstaben, man schreibt also zum Beispiel in der \LaTeX"=Quelle:
\verb=Auf"|lage=.

%.............................................................................
\subsection{Paket \paket{microtype}}

Dieses Paket sorgt tendenziell für schöneres Aussehen der Seiten; (siehe die
Schlagwörter "`\emph{protrusion}"' und "`\emph{expansion}"' in der
Dokumentation). Einfach verwenden.

%.............................................................................
\subsection{Paket \paket{amsmath}}

Dieses Paket ist unter anderem dann nützlich, wenn freigestellte
(\emph{display math}) Formeln zu setzen hat, die mehrere Zeilen benötigen.
Hier zeigen wir einfach zwei Beispiele, für genauere Informationen konsultiere
man die Dokumentation. 

\begin{tcblisting}{}
  \begin{align*}
    (x+y)^2 &= (x+y) (x+y) \\
            &= x(x+y) + y(x+y) \\
            &= x^2 + xy + yx + y^2 \\
            &= x^2 + 2xy + y^2 \\
  \end{align*}
\end{tcblisting}

Im zweiten Beispiel wird auch noch das Kommando
\verb|\text{|\meta{Text}\verb|}| benutzt, um in Formeln normal gesetzten Text
einzufügen.

\begin{tcblisting}{}
  \begin{alignat*}{2}
    (x+y)^2 
    &= (x+y) (x+y)          \\
    &= x(x+y) + y(x+y)      &\qquad& \text{Distributivität} \\
    &= x^2 + xy + yx + y^2  && \text{nochmal Distributivität}\\
    &= x^2 + 2xy + y^2      && \text{Kommutativität}\\
  \end{alignat*}
\end{tcblisting}

%.............................................................................
\subsection{Paket \paket{enumitem}}

Dieses Paket erlaubt es, bequem den für meinen Geschmack zu großzügig
bemessenen Platz in Listen zu reduzieren.
%
Das wird durch das Kommande \verb|\setlist{noitemsep}| in der Präambel
eingestellt.

%-----------------------------------------------------------------------------
\section{No Go}
\label{sec:no-go}

\paragraph{Layout von Absätzen.} \emph{Niemals} 
%
ändere man in der Präambel die Werte von \verb|\parskip| und
\verb|\parindent|.

\paragraph{Text in Formeln.} \emph{Niemals}
%
schreibe man normalen Text in Formeln \emph{ohne} Benutzung von
\verb|\text{}|: \verb|$Text$| liefert $Text$, das ist völliger Murks.
Vernünftig sind
\begin{itemize}
\item \verb|$\text{Text}$| $\to$ $\text{Text}$
\item \verb|$\mathit{Text}$| $\to$ $\mathit{Text}$
\end{itemize}


\paragraph{Fußnoten.} \emph{Niemals}
%
übertreibe man es mit den Fußnoten.
%
Am besten lässt man es ganz.
%
Das Zitat "`Don't use footnotes in your books, Don."' von Knuths Frau zitiert
er natürlich in einer Fußnote (im \TeX book) \dots

%-----------------------------------------------------------------------------
\section{Weitere Tipps}
\label{sec:tipps}

Kommentare in \LaTeX{} werden mit einem Prozentzeichen \verb|%| eingeleitet
und reichen bis zum Zeilenende.
% 
Genauer gesagt wird auch noch das Zeilenendezeichen mit verschluckt, sowie
alle Leerzeichen in der darauf"|folgenden Zeile.

Wir haben und angewöhnt, wie in dieser \LaTeX"=Quelle vorgemacht, aufeinander
folgende Sätze durch eine Zeile zu trennen, in der einfach in der ersten
Spalte das Kommentarzeichen \verb|%| steht.
%
Die bewirken inhaltlich für \LaTeX{} gar nichts.
%
Außerdem wird darauf geachtet, dass die Zeilen eine "`überschaubare'' Länge
haben.
%
Diese beiden Maßnahmen haben zur Folge, dass sich bei Änderungen eines Satzes
nur eine kleine Menge von Zeilen der Eingabedatei ändert.
%
Wenn man sich mit \prgname{diff} oder einem ähnlichen Programm die
Unterschiede \zB zwischen der aktuellen und einer früheren Version ansieht
(\prgname{git}, \prgname{mercurial} und Co.~lassen grüßen), erstrecken sich
die Änderungen nur über wenige Zeilen.
%
Das findet der Autor dieser Zeilen sehr nützlich.

% -----------------------------------------------------------------------------
\section{Literatur(verzeichnis)}
\label{sec:literatur}

Inhaltlich geht es in diesem Abschnitt um Literatur und
Literaturverzeichnisse.
%
Die \LaTeX"=Quelle zeigt aber auch, wie man Abbildungen aufnimmt, die so groß
sind, dass man nicht "`zu Fuß"' festlegen will, an welcher Stelle sie im
Pdf"=Dokument auftauchen.
%
Und man sieht, wie man auf solche Abbildungen verweist.

Hier sind beispielhaft zwei \emph{bibtex entries}, nämlich in
Abbildung~\ref{bibtex:article} einer für einen Zeitschriftenaufsatz und in
Abbildung~\ref{bibtex:inproceedings} einer für einen Konferenzbeitrag.

\begin{figure}[htb]
\begin{tcblisting}{listing only}
@article{Worsch_2009_AUC_ar,
  author  = {Thomas Worsch and Hidenosuke Nishio},
  title   = {Achieving universality of {CA} by changing the neighborhood},
  journal = {Journal of Cellular Automata},
  year    = {2009},
  volume  = {4},
  number  = {3},
  pages   = {237--246},
}
\end{tcblisting}
\caption{Ein \emph{bibtex entry} für einen Zeitschriftenaufsatz}
\label{bibtex:article}
\end{figure}

\begin{figure}[htb]
\begin{tcblisting}{listing only}
@inproceedings{Worsch_2012_IUA_ip_acri,
  author    = {Thomas Worsch},
  title     = {({I}ntrinsically?) Universal Asynchronous Cellular Automata},
  editor    = {Georgios Sirakoulis and Stefania Bandini},
  booktitle = {Proceedings ACRI 2012},
  year      = {2012},
  pages     = {689--698},
  publisher = {Springer},
  series    = {LNCS},
  volume    = {7495},
}
\end{tcblisting}
\caption{Ein \emph{bibtex entry} für einen Konferenzbeitrag}
\label{bibtex:inproceedings}
\end{figure}

Wozu das gut ist, erläutern wir ein andermal \dots

% -----------------------------------------------------------------------------
\section{Was noch fehlt}
\label{sec:todo}

\begin{itemize}[noitemsep]
\item Erläuterungen zum Paket \paket{graphicx}
\item Erläuterungen zum Paket \paket{booktabs}
\item Erläuterungen zum Erstellen von Bildern: Paket \paket{tikz}
\item Literatur
  \begin{itemize}
  \item Wie zitiert man?
  \item Erläuterungen zur automatischen Erstellung eines Literaturverzeichnisses
  \end{itemize}
\item allgemein: schöneres Aussehen
  \begin{itemize}
  \item Auswahl anderer Schriften?
  \end{itemize}
\end{itemize}
%
Was möchten Sie noch wissen?

\end{document}
%=============================================================================
%%% Der Rest ist für meine Editor  (GNU Emacs, was sonst ;-) :
%%%
%%% Local Variables:
%%% TeX-command-default: "XPDFLaTeX"
%%% fill-column: 78
%%% TeX-master: t
%%% End:
