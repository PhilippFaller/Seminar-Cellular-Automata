\documentclass[11pt]{article}

\usepackage{fontspec}
\usepackage[ngerman]{babel}
\usepackage[utf8]{inputenc}

\usepackage{amsmath}
\usepackage{mathtools}
\usepackage[standard, thref]{ntheorem}

\usepackage[tt=false]{libertine}   % !!!!! das muss man nicht nutzen
\usepackage[libertine]{newtxmath}  % !!!!! das muss man nicht nutzen
\usepackage[supstfm=libertinesups,supscaled=1.2,raised=-.13em]{superiors} % params taken from doc

%
%\usepackage{tgpagella}
%\usepackage[euler-digits]{eulervm}
%
\usepackage{microtype}

\usepackage{fancyvrb}

%\usepackage{graphicx}

\usepackage{booktabs}
\usepackage[shortlabels]{enumitem}
\setlist{noitemsep}

\usepackage{titlesec}
%\usepackage{tcolorbox}
%\tcbuselibrary{listingsutf8}

\usepackage{bbold}
\newcommand{\Z}{\mathbb{Z}}

\usepackage{csquotes}


%-----------------------------------------------------------------------------
% für das Deckblatt

\usepackage{tikz}
\usetikzlibrary{automata}

\newcommand{\teilnehmername}{Philipp Faller} % !!!!!
\newcommand{\teilnehmermatrnr}{1939704}        % !!!!!
\newcommand{\seminarart}{Proseminar}           % !!!!!  oder Seminar
\newcommand{\seminarlp}{3 LP}                  % !!!!!  Prosem immer 3 LP, Sem 3 oder 4 LP
\newcommand{\seminarjahr}{2017}                % !!!!!
%-----------------------------------------------------------------------------
\newcommand{\meta}[1]{$\langle$\textit{#1}$\rangle$}
\newcommand{\paket}[1]{\texttt{#1}}
\newcommand{\prgname}[1]{\texttt{#1}}
\newcommand{\defWord}[1]{\emph{#1}}

%-----------------------------------------------------------------------------
\author{Thomas Worsch}
\title{Hinweise zu Seminarausarbeitungen}

%=============================================================================
\begin{document}
%=======================================================================
% Anfang erste Seite
{\thispagestyle{empty}\large\sffamily\raggedright
%
\begin{tikzpicture}[remember picture,overlay]
  \coordinate[xshift=5mm,yshift=-5mm] (NW) at (current page.north west) {};
  \coordinate[xshift=-5mm,yshift=-5mm] (NE) at (current page.north east) {};
  \coordinate[xshift=-5mm,yshift=13mm] (SE) at (current page.south east) {};
  \coordinate[xshift=5mm,yshift=13mm] (SW) at (current page.south west) {};

  \draw[line width=0.25pt] (NW)
    [rounded corners=5mm] -- (NE) 
    [sharp corners] -- (SE)
    [rounded corners=5mm] -- (SW)
    [sharp corners] -- cycle
  ;
\end{tikzpicture}
%
\unskip % keine Ahnung warum das nötig ist
\noindent \textbf{\Large \seminarart\ (\seminarlp)} 
\\[\baselineskip]
%
Zellularautomaten und diskrete komplexe Systeme
% für Fortgeschrittene  % nur für das 4 Leistungspunkte Seminar !!!!!
\\[1ex]
%
im Sommersemester \seminarjahr

\vspace*{3\baselineskip}

\noindent \textbf{\Large Ausarbeitung} \\[\baselineskip]
%
von \textbf{\teilnehmername}, Matr.nr.~\teilnehmermatrnr

\vspace*{3\baselineskip}

\noindent \textbf{\Large Thema} \\[\baselineskip]
%
% nachfolgende ein Beispiel, für Konferenzbeiträge, Buchausschnitte, ...
% bitte analog vorgehen !!!!!
%
Angela Wu und Azriel Rosenfeld (1979)\\[1ex]
%
\textit{Cellular Graph Automata I }\\[1ex]
%
Information and Controll, Band \textbf{42}, S.~305-329
}
\clearpage
% Ende erste Seite
%=======================================================================
% Anfang zweite Seite
{\thispagestyle{empty}\raggedright

\noindent \textbf{\Large Erklärung}\\[1ex]
gemäß \S 6 (11) der Prüfungsordnung Informatik % !!!!! oder \S 6 (7) (bei MasterPO 2015)
(Bachelor) % oder Master !!!!!
\\[\baselineskip]

\noindent
Ich versichere wahrheitsgemäß, die Seminarausarbeitung zum
\seminarart{} "`Zellularautomaten und diskrete komplexe Systeme"' im
Sommersemester \seminarjahr{} selbstständig angefertigt, alle
benutzten Hilfsmittel vollständig und genau angegeben und alles
kenntlich gemacht zu haben, was aus Arbeiten anderer unverändert oder
mit Abänderungen entnommen wurde.

\vspace*{30mm}
\noindent
\begin{tabular}{@{}l}
  \hline
   \\[-1ex]
  \hbox to 0.6\textwidth{(\teilnehmername, Matr.nr.~\teilnehmermatrnr) \hss}
\end{tabular}
}
\clearpage
% Ende zweite Seite
%=======================================================================

%-----------------------------------------------------------------------------

\section{Übersicht}

\section{$d$"=Graphen}
\begin{definition}[$d$=Graph]
Sei $L$ eine endliche, nicht leere Menge von Beschriftungen, mit einem ausgezeichneten Element $\#$. 
Ein \defWord{$d$-Graph} über $L$ ist ein 4"=Tupel $\Gamma = \left(N, A, f, g\right)$, wobei
\begin{itemize}
	\item[$N$] eine endliche, nicht leere Menge aus Knoten ist.
	\item[$A$] $\subseteq N \times N$ ist eine symmetrische Relation über $N$ und heißt \defWord{Kantenmenge} von $L$.
	\item[$f$] $: N \rightarrow L$ ist eine Abbildung, sodass für $\left(m, n\right) \in A$ gilt: $f\left(m\right) = \# \implies f\left(n\right) \neq \#$. $f$  heißt \defWord{Beschriftungsfunktion}.
	\item[$g$] $: A \rightarrow Z_d$ ist eine Abbildung, wobei $Z_d = \left \{1, 2,\text{\dots}, d \right \}$, die folgende Eigenschaften besitzt:
	
	Sei $A_n := \left \{\left(n, m\right) \in A\right \}$ die Menge aller Kanten, die von einem Knoten $n \in N$ ausgehen.
	\begin{enumerate}
		\item Falls $f\left(n\right) \neq \#$, ist $\left|A_n\right| = d$ und $g \vert_{A_n}$ ($g$ eingeschränkt auf $A_n$) ist eine Bijektion zwischen $A_n$ und $Z_d$.
		\item Falls $f\left(n\right) = \#$, ist $\left|A_n\right| = 1$
	\end{enumerate}
	
\end{itemize}
\end{definition}

Informell ist $\Gamma$ also ein Graph, dessen Knoten mit Beschriftungen aus $L$ versehen sind. 
Jede Kante hat eine Nummer aus $\left\{1, \dots, d\right\}$. 
Dabei gilt im Allgemeinen $g\left(n, m\right) \neq g\left(m, n\right)$. 
Jeder Knoten, dessen Beschriftung nicht $\#$ ist, hat Grad $d$, jeder andere hat Grad $1$. 

\begin{beispiel}
	\label{dGraph}
	\begin{tikzpicture}[node distance=2cm]
	\node[state](a){$a$};
	\node[state](c)[below right of=a]{$c$};
	\node[state](b)[above right of=c]{$b$};
	\foreach \q/\p in {a/b, a/c, b/c}
			\draw[->] (\p) edge[bend left = 9] (\q)
					  (\q) edge[bend left = 9] (\p);
						
	\end{tikzpicture}
\end{beispiel}

\begin{beispiel}
	\begin{tikzpicture}[node distance=2cm]
		\node[state](a){$a$};
		\node[state](b)[right of = a]{$b$};
		\node[state](d)[right of = b]{$d$};
		\node[state](f)[below right of = d]{$f$};
		\node[state](e)[below left of = f]{$e$};
		\node[state](c)[left of = e]{$c$};
		\node[state](a1)[left of = a]{$\#$};
		\node[state](a2)[below left of = a]{$\#$};
		\node[state](c1)[left of = c]{$\#$};
		\node[state](f1)[above right of = f]{$\#$};
		\node[state](f2)[below right of = f]{$\#$};
		
		
		\foreach \p/\q in {a/b, b/d, b/c, d/e, d/f, e/f, c/e, a1/a, a2/a, c1/c, f1/f, f2/f}
			\draw[->] 
				(\q) edge[bend left=9] (\p)
				(\p) edge[bend left=9] (\q) 			
		;
	\end{tikzpicture}
\end{beispiel}

\begin{definition}[Zugrunde liegender Graph]
	Sei $\Gamma$ ein $d$"=Graph. 
	Der \defWord{$\Gamma$ zugrunde liegende Graph $G = \left(V, E\right)$} ist der Graph, mit der Knotenmenge $G = N\setminus \left\{n \in N \mid f\left(n\right) = \# \right\}$ und der Kantenmenge  $E = A\setminus \left\{\left(n, m\right) \in A \mid f\left(n\right) = \# \lor f\left(m\right) = \# \right\}$. 
	$G$ ist also der Graph, der durch entfernen der Knoten, die mit $\#$ beschriftet sind, entsteht.

	Ein $d$"=Graph heißt \defWord{zusammenhängend}, wenn der zugrunde liegende Graph zusammenhängend ist.
\end{definition}
\begin{definition}[Nachbarschaft]
	Sei $\left(n, m\right) \in A$ und $g\left(n, m\right) = i$. 
	Dann heißt $m$ der \defWord{$i$"=te Nachbar von $n$}.
	Da $A$ symmetrisch ist impliziert das, dass $\left(m, n\right) \in A$ und $\exists j \in Z_d : g\left(m, n\right) = j$.
	Dann ist $n$ der $j$"=te Nachbar von $m$. 
	Definiere die Abbildung $h : N \times Z_d \rightarrow N$, so dass $h\left(n, i\right) = m \iff g\left(n, m\right) = i$.
\end{definition}

\section{Zellulare $d$"=Graph Automaten und Akzeptoren}

\begin{definition}[$d$"=Graph Automat]
	Ein \defWord{Zellulärer $d$"=Graph Automat} $\mathcal{M}$ ist ein Tripel $\left(\Gamma, M, H\right)$, wobei
	\begin{itemize}
		\item[$\Gamma$] ein $d$"=Graph $\left(N, A, f, g\right)$ über einer Menge von Beschriftungen $L$ ist.
		\item[$M$]  ist ein endlicher Automat $\left(Q, \delta\right)$, mit 
		\begin{itemize}
			\item[$Q$] ist eine nicht leere Menge von Zuständen, mit $L \subseteq Q$.
			\item[$\delta$] $: Q \times Z_d^d \times Q^d \rightarrow \mathcal{P}\left(Q\right)$ ist eine Abbildung, für die gilt $\delta \left(\left \{\left(\#, z, q\right) \vert z \in Z_d^d, q \in Q^d\right \}\right) = \left \{\# \right \}$. $\delta$ heißt \defWord{Zustandsübergangsfunktion}.
		\end{itemize}
		\item[$H$] $: N \rightarrow Z_d^d$ ist eine Abbildung. 
		Für $n \in N$ heißt das Bild $H \left(n\right) = \left(t_1, \text{\dots}, t_n\right) \in Z_d^d$ \defWord{Nachbarschaftsvektor} von $n$. Es gilt $\forall i \in Z_d : H\left(n\right)_i = \begin{dcases}
		g\left(h\left(n, i\right), n\right) & ,f\left(n\right) \neq \# \\
		t & , f\left(n\right) = \# \\
		\end{dcases}$.
	\end{itemize}
\end{definition}

Informell ist jeder Knoten $n$ in $\Gamma$ ein endlicher Automat $M_n$, mit Startzustand $f\left(n\right) \in L$. 
(Wir sagen auch $\Gamma$ ist der \defWord{Eingabegraph} von $M$.) 
In jedem Zeitschritt liest $M_n$ die Zustände seiner Nachbarn, sowie den Nachbarschaftsvektor und wechselt dann, wie durch die Zustandsübergangsfunktion $\delta$ festgelegt, in einen anderen Zustand. 
Der Nachbarschaftsvektor sagt einem Knoten dabei, der wievielte Nachbar er selbst von seinen Nachbarn ist. 
In \thref{dGraph} ist zum Beispiel $H\left(a\right) = \left(2, 1\right)$. 
Das erlaubt einem Automaten $M_n$ den Zustand eines Nachbarn zu in Abhängigkeit davon zu betrachten, der wievielte Nachbar er von diesem Knoten ist.

\begin{definition}[Konfiguration]
	Eine \defWord{Konfiguration von $\mathcal{M}$} ist eine Abbildung $c : N \rightarrow Q$, die jedem Knoten einen Zustand zuordnet.
	Sei $n$ ein Knoten mit Zustand $q = c\left(n\right)$ und Nachbarn $m_1, \dots, m_d$. 
	Sei $q_m = \left( c\left( m_1\right), \dots , c\left(m_d\right)\right) \in Q^d$ der Vektor, der die Zustände der Nachbarn von $n$ enthält. 
	Der Folgezustand $c'\left( n\right)$ des Knotens $n$ ist dann gegeben durch $c'\left(n\right) = \delta\left(q, H\left(n\right),q_m \right) $. 
	Wir sagen $\delta$ überführt $c$ in $c'$ und schreiben $c \vdash c'$.
	Wird bei der Überführung $H$ nicht benutzt, das heißt $\delta$ bildet effektiv von $Q^{d+1}$ nach $Q$ ab, so heißt $\mathcal{M}$ \defWord{Schwacher Zellularer $d$"=Graph Automat}.
\end{definition}

\begin{definition}[Akzeptanz]
	Ein Zellularer $d$"=Graph Automat, bei dem genau ein Knoten eine ausgezeichnete Marke als Teil seines Zustandes  besitzt, heißt \defWord{Zellularer d"=Graph Automat mit ausgezeichnetem Knoten}.
	
	Ein \defWord{Zellularer $d$"=Graph Akzeptor} ist ein Zellularer $d$"=Graph Automat $\mathcal{M} = \left(\Gamma, M, H \right)$ mit einem ausgezeichneten Knoten, sodass $\mathcal{M}$ ein endlicher Akzeptor $\left(L, Q, \delta, F\right)$, mit Startzuständen $L$, Zustandsmenge $Q$, Zustandsüberführungsfunktion $\delta$ und akzeptierenden Zuständen $F \subseteq Q$.
	
	Eine \defWord{Startkonfiguration} von $\mathcal{M}$ ist eine Abbildung $c : N \rightarrow L$. 
	Eine \defWord{Endkonfiguration} ist eine Abbildung $c' : N \rightarrow Q$, bei der für den ausgezeichneten Knoten $\hat{n}$ gilt $c'\left(n\right) \in F$. 
	Also eine Konfiguration, bei der der ausgezeichnete Knoten sich in einem akzeptierenden Zustand befindet.
	
	Ein Zellularer $d$"=Graph Akzeptor \defWord{akzeptiert} den $d$"=Graph $\Gamma$, wenn es eine endliche Folge von Konfigurationen $\left(c_i\right)_{1 \le i \le m}$ gibt, mit einer Startkonfiguration $c_0 = f$, einer Endkonfiguration $c_m$ und $c_i \vdash c_{i+1}$ für $1 \le i \le m$.
\end{definition}

\begin{definition}[Sprachen]
	Für einen gegebenen endlichen Akzeptor $M = \left(L, Q, \delta, F\right)$ sei $\mathcal{C} \left(M\right) = \left\{\mathcal{M} = \left(\Gamma, M, H\right) \mid \mathcal{M} \text{ ist ein zellularer $d$"=Graph Akzeptor mit ausgezeichnetem Knoten} \right\}$  die \defWord{Klasse, der durch $M$ festgelegten Zellularen Graph Akzeptoren}. 
	Wenn keine Verwechslungsgefahr besteht, nennen wir $\mathcal{C}\left(M\right)$ auch den \defWord{Zellularen $d$"=Graph Akzeptor von $M$}.
	
	Die \defWord{Sprache von $d$"=Graphen, die durch $\mathcal{C}\left(M\right)$ akzeptiert wird} ist die Menge $\mathcal{L}\left(M\right) = \left\{\Gamma \mid \mathcal{M} = \left(\Gamma, M, H\right) \in \mathcal{C}\left(M\right) \text{ akzeptiert } \Gamma \right\}$
	Ein $d$"=Graph $\Gamma$ wird von $\mathcal{C}$ \defWord{akzeptiert} $:\iff \Gamma \in \mathcal{L}\left(M\right)$.
	
	Die Klasse aller Sprachen, die von einem (deterministischen) Zellularen $d$"=Graph Akzeptor akzeptiert wird, heißt die Klasse der (deterministischen) \defWord{$d$"=Graph Sprachen} $(D)\mathcal{C}d\mathcal{L}$. 
\end{definition}

\begin{definition}[$d$"=Graph Entscheider]
	Wenn $M$ eine weiter Teilmenge $R \subseteq Q$ mit $R \cap F = \emptyset$ besitzt, nennen wir diese die \defWord{ablehnenden Zustände} von $M = \left(L, Q, \delta, F, R\right)$ und schreiben $M$ als 5-Tupel.
	
	$M$ \defWord{lehnt $\Gamma$ ab}, wenn der Automat $M_n$ am ausgezeichneten Knoten $n$ nach endlich vielen Schritten in einem ablehnenden Zustand ist.
	Sei  $\mathcal{L}'\left(M\right) = \left\{\Gamma \mid \mathcal{M} = \left(\Gamma, M, H\right) \in \mathcal{C}\left(M\right) \text{ lehnt } \Gamma \text{ ab}\right\}$.
	
	Ein $d$"=Graph $\Gamma$ wird von $\mathcal{C}$ \defWord{abgelehnt} $:\iff \Gamma \in \mathcal{L}'\left(M\right)$.
	
	$\mathcal{C}\left(M\right)$ \defWord{entscheidet} eine $d$"=Graphen Sprache $\mathcal{L} \iff \mathcal{C}$ akzeptiert alle $d$"=Graphen $\Gamma \in \mathcal{L} \land \mathcal{C}\left(M\right)$ lehnt alle $d$"=Graphen $\Gamma \notin \mathcal{L}$ ab.
	
	$\mathcal{L}$ wird ein \defWord{Zellulares $d$"=Graph Prädikat} und $\mathcal{C}\left(M\right)$ ein \defWord{Zellularer $d$"=Graph Entscheider} genannt. 
	Jedes Zellulare $d$"=Graph Prädikat ist insbesondere eine $d$"=Graph Sprache.
\end{definition}

\begin{definition}
	Für einen $d$"=Graphen $\Gamma$ bezeichen $U\left(\Gamma\right)$ den $\Gamma$ zugrunde liegenden Graphen. 
	Definiere weiter $\mathcal{G}_L^d = \left\{\gamma \mid \gamma = U\left(\Gamma\right) \text{ für einen $d$"=Graph } \Gamma \text{ über } L\right\}$, sowie für $\gamma \in \mathcal{G}_L^d : U^{-1}\left(\gamma\right) = \left\{\Gamma \mid \gamma = U\left(\Gamma\right)\right\}$. 
\end{definition}
\section{Graphen Eigenschaften}

\section{Spannbaum durch Breitensuche}

\begin{beispiel}
	\begin{tikzpicture}[node distance=2cm]
		\node[state](a){$a\mid ?$};
		\node[state, fill=red, text=white, accepting](b)[right of = a]{$b$};
		\node[state](d)[right of = b]{$d\mid ?$};
		\node[state](f)[below right of = d]{$f\mid ?$};
		\node[state](e)[below left of = f]{$e\mid ?$};
		\node[state](c)[left of = e]{$c\mid ?$};
		\node[state](a1)[left of = a]{$\#$};
		\node[state](a2)[below left of = a]{$\#$};
		\node[state](c1)[left of = c]{$\#$};
		\node[state](f1)[above right of = f]{$\#$};
		\node[state](f2)[below right of = f]{$\#$};
		
		\foreach \p/\q in {a/b, b/d, b/c, d/e, d/f, e/f, c/e, a1/a, a2/a, c1/c, f1/f, f2/f}
			\draw[->] 
				(\q) edge[bend left=9] (\p)
				(\p) edge[bend left=9] (\q) 			
		;
		\end{tikzpicture}
		
		\begin{tikzpicture}[node distance=2cm]
		\node[state, fill=red, draw=none, text=white](a){$a\mid b$};
		\node[state, fill=blue, text=white, accepting](b)[right of = a]{$b$};
		\node[state, fill=red, draw=none, text=white](d)[right of = b]{$d\mid b$};
		\node[state](f)[below right of = d]{$f\mid ?$};
		\node[state](e)[below left of = f]{$e\mid ?$};
		\node[state, fill=red, draw=none, text=white](c)[left of = e]{$c\mid b$};
		\node[state](a1)[left of = a]{$\#$};
		\node[state](a2)[below left of = a]{$\#$};
		\node[state](c1)[left of = c]{$\#$};
		\node[state](f1)[above right of = f]{$\#$};
		\node[state](f2)[below right of = f]{$\#$};
		
		\foreach \p/\q in {a/b, b/d, b/c, d/e, d/f, e/f, c/e, a1/a, a2/a, c1/c, f1/f, f2/f}
			\draw[->] 
				(\q) edge[bend left=9] (\p)
				(\p) edge[bend left=9] (\q) 			
		;
		\end{tikzpicture}
		\newline
		\begin{tikzpicture}[node distance=2cm]
		\node[state, fill=yellow, draw=none](a){$a\mid b$};
		\node[state, fill=blue, text=white, accepting](b)[right of = a]{$b$};
		\node[state, fill=blue, draw=none, text=white](d)[right of = b]{$d\mid b$};
		\node[state, fill=red, draw=none, text=white](f)[below right of = d]{$f\mid d$};
		\node[state, fill=red, draw=none, text=white](e)[below left of = f]{$e\mid d$};
		\node[state, fill=blue, draw=none, text=white](c)[left of = e]{$c\mid b$};
		\node[state](a1)[left of = a]{$\#$};
		\node[state](a2)[below left of = a]{$\#$};
		\node[state](c1)[left of = c]{$\#$};
		\node[state](f1)[above right of = f]{$\#$};
		\node[state](f2)[below right of = f]{$\#$};
		
		\foreach \p/\q in {a/b, b/d, b/c, d/e, d/f, e/f, c/e, a1/a, a2/a, c1/c, f1/f, f2/f}
			\draw[->] 
				(\q) edge[bend left=9] (\p)
				(\p) edge[bend left=9] (\q) 			
		;
		\end{tikzpicture}
		\newline
		\begin{tikzpicture}[node distance=2cm]
		\node[state, fill=yellow, draw=none](a){$a\mid b$};
		\node[state, fill=blue, text=white, accepting](b)[right of = a]{$b$};
		\node[state, fill=yellow, draw=none](d)[right of = b]{$d\mid b$};
		\node[state, fill=yellow, draw=none](f)[below right of = d]{$f\mid d$};
		\node[state, fill=yellow, draw=none](e)[below left of = f]{$e\mid d$};
		\node[state, fill=yellow, draw=none](c)[left of = e]{$c\mid b$};
		\node[state](a1)[left of = a]{$\#$};
		\node[state](a2)[below left of = a]{$\#$};
		\node[state](c1)[left of = c]{$\#$};
		\node[state](f1)[above right of = f]{$\#$};
		\node[state](f2)[below right of = f]{$\#$};
		
		\foreach \p/\q in {a/b, b/d, b/c, d/e, d/f, e/f, c/e, a1/a, a2/a, c1/c, f1/f, f2/f}
		\draw[->] 
		(\q) edge[bend left=9] (\p)
		(\p) edge[bend left=9] (\q) 			
		;
		\end{tikzpicture}
		\newline
		\begin{tikzpicture}[node distance=2cm]
		\node[state, fill=yellow, draw=none](a){$a\mid b$};
		\node[state, fill=yellow, accepting](b)[right of = a]{$b$};
		\node[state, fill=yellow, draw=none](d)[right of = b]{$d\mid b$};
		\node[state, fill=yellow, draw=none](f)[below right of = d]{$f\mid d$};
		\node[state, fill=yellow, draw=none](e)[below left of = f]{$e\mid d$};
		\node[state, fill=yellow, draw=none](c)[left of = e]{$c\mid b$};
		\node[state](a1)[left of = a]{$\#$};
		\node[state](a2)[below left of = a]{$\#$};
		\node[state](c1)[left of = c]{$\#$};
		\node[state](f1)[above right of = f]{$\#$};
		\node[state](f2)[below right of = f]{$\#$};
		
		\foreach \p/\q in {a/b, b/d, b/c, d/e, d/f, e/f, c/e, a1/a, a2/a, c1/c, f1/f, f2/f}
		\draw[->] 
		(\q) edge[bend left=9] (\p)
		(\p) edge[bend left=9] (\q) 			
		;
		\end{tikzpicture}
		\newline
		\begin{tikzpicture}[node distance=2cm]
		\node[state, accepting](b){$b$};
		\node[state](c)[below of = b]{$c\mid b$};
		\node[state](d)[right of = c]{$d\mid b$};
		\node[state](a)[left of = c]{$a\mid b$};
		\node[state](a1)[below of = a]{$\#$};
		\node[state](a2)[left of = a1]{$\#$};
		\node[state](c1)[below of = c]{$\#$};
		\node[state](e)[right of = c1]{$e\mid d$};
		\node[state](f)[right of = e]{$f\mid d$};
		\node[state](f1)[below left of = f]{$\#$};
		\node[state](f2)[below right of = f]{$\#$};
		
		\foreach \p/\q in {a/b, b/d, b/c, d/e, d/f, a1/a, a2/a, c1/c, f1/f, f2/f}
		\draw[->] 
		(\q) edge[bend left=9] (\p)
		(\p) edge[bend left=9] (\q) 			
		;
		\end{tikzpicture}
		
\end{beispiel}

\section{Radius bestimmen und Label finden}

\begin{beispiel}
	\begin{tikzpicture}[node distance=2cm]
		\node[state, fill=red, text=white, accepting] (a) {$0\mid 1$};
		\node[state] (b)[left of = a] {$0\mid 0$};
		\node[state] (c)[left of = b] {$0\mid 0$};
		\node[state] (d)[left of = c] {$0\mid 0$};
		\node[state] (e)[left of = d] {$0\mid 0$};
		\node[state] (f)[left of = e] {$0\mid 0$};
		
		\foreach \p/\q in {a/b, b/c, c/d, d/e, e/f}
			\draw[->] 
				(\q) edge (\p)
				(\p) edge (\q) 			
		;
	\end{tikzpicture}
	
	\begin{tikzpicture}[node distance=2cm]
	\node[state, accepting] (a) {$1\mid 0$};
	\node[state, fill=red, text=white, draw = none] (b)[left of = a] {$0\mid 0$};
	\node[state] (c)[left of = b] {$0\mid 0$};
	\node[state] (d)[left of = c] {$0\mid 0$};
	\node[state] (e)[left of = d] {$0\mid 0$};
	\node[state] (f)[left of = e] {$0\mid 0$};
	
	\foreach \p/\q in {a/b, b/c, c/d, d/e, e/f}
	\draw[->] 
	(\q) edge (\p)
	(\p) edge (\q) 			
	;
	\end{tikzpicture}
	
	\begin{tikzpicture}[node distance=2cm]
	\node[state] (a) {$1\mid 1$};
	\node[state] (b)[left of = a] {$0\mid 0$};
	\node[state, fill=red, text=white, draw = none] (c)[left of = b] {$0\mid 0$};
	\node[state] (d)[left of = c] {$0\mid 0$};
	\node[state] (e)[left of = d] {$0\mid 0$};
	\node[state] (f)[left of = e] {$0\mid 0$};
	
	\foreach \p/\q in {a/b, b/c, c/d, d/e, e/f}
	\draw[->] 
	(\q) edge (\p)
	(\p) edge (\q) 			
	;
	\end{tikzpicture}
	
	\begin{tikzpicture}[node distance=2cm]
	\node[state] (a) {$0\mid 0$};
	\node[state] (b)[left of = a] {$0\mid 1$};
	\node[state] (c)[left of = b] {$0\mid 0$};
	\node[state, fill=red, text=white, draw = none] (d)[left of = c] {$0\mid 0$};
	\node[state] (e)[left of = d] {$0\mid 0$};
	\node[state] (f)[left of = e] {$0\mid 0$};
	
	\foreach \p/\q in {a/b, b/c, c/d, d/e, e/f}
	\draw[->] 
	(\q) edge (\p)
	(\p) edge (\q) 			
	;
	\end{tikzpicture}
	
	\begin{tikzpicture}[node distance=2cm]
	\node[state] (a) {$0\mid 1$};
	\node[state] (b)[left of = a] {$1\mid 0$};
	\node[state] (c)[left of = b] {$0\mid 0$};
	\node[state] (d)[left of = c] {$0\mid 0$};
	\node[state, fill=red, text=white, draw = none] (e)[left of = d] {$0\mid 0$};
	\node[state] (f)[left of = e] {$0\mid 0$};
	
	\foreach \p/\q in {a/b, b/c, c/d, d/e, e/f}
	\draw[->] 
	(\q) edge (\p)
	(\p) edge (\q) 			
	;
	\end{tikzpicture}
	
	\begin{tikzpicture}[node distance=2cm]
	\node[state] (a) {$1\mid 0$};
	\node[state] (b)[left of = a] {$1\mid 0$};
	\node[state] (c)[left of = b] {$0\mid 0$};
	\node[state] (d)[left of = c] {$0\mid 0$};
	\node[state] (e)[left of = d] {$0\mid 0$};
	\node[state, fill=red, text=white, draw = none] (f)[left of = e] {$0\mid 0$};
	
	\foreach \p/\q in {a/b, b/c, c/d, d/e, e/f}
	\draw[->] 
	(\q) edge (\p)
	(\p) edge (\q) 			
	;
	\end{tikzpicture}
	
	\begin{tikzpicture}[node distance=2cm]
	\node[state] (a) {$1\mid 1$};
	\node[state] (b)[left of = a] {$1\mid 0$};
	\node[state] (c)[left of = b] {$0\mid 0$};
	\node[state] (d)[left of = c] {$0\mid 0$};
	\node[state, fill=red, text=white, draw = none] (e)[left of = d] {$0\mid 0$};
	\node[state] (f)[left of = e] {$0\mid 0$};
	
	\foreach \p/\q in {a/b, b/c, c/d, d/e, e/f}
	\draw[->] 
	(\q) edge (\p)
	(\p) edge (\q) 			
	;
	\end{tikzpicture}
	
	\begin{tikzpicture}[node distance=2cm]
	\node[state] (a) {$0\mid 0$};
	\node[state] (b)[left of = a] {$1\mid 1$};
	\node[state] (c)[left of = b] {$0\mid 0$};
	\node[state, fill=red, text=white, draw = none] (d)[left of = c] {$0\mid 0$};
	\node[state] (e)[left of = d] {$0\mid 0$};
	\node[state] (f)[left of = e] {$0\mid 0$};
	
	\foreach \p/\q in {a/b, b/c, c/d, d/e, e/f}
	\draw[->] 
	(\q) edge (\p)
	(\p) edge (\q) 			
	;
	\end{tikzpicture}
	
	\begin{tikzpicture}[node distance=2cm]
	\node[state] (a) {$0\mid 1$};
	\node[state] (b)[left of = a] {$0\mid 0$};
	\node[state, fill=red, text=white, draw = none] (c)[left of = b] {$0\mid 1$};
	\node[state] (d)[left of = c] {$0\mid 0$};
	\node[state] (e)[left of = d] {$0\mid 0$};
	\node[state] (f)[left of = e] {$0\mid 0$};
	
	\foreach \p/\q in {a/b, b/c, c/d, d/e, e/f}
	\draw[->] 
	(\q) edge (\p)
	(\p) edge (\q) 			
	;
	\end{tikzpicture}
	
	\begin{tikzpicture}[node distance=2cm]
	\node[state] (a) {$1\mid 0$};
	\node[state, fill=red, text=white, draw = none] (b)[left of = a] {$0\mid 0$};
	\node[state] (c)[left of = b] {$1\mid 0$};
	\node[state] (d)[left of = c] {$0\mid 0$};
	\node[state] (e)[left of = d] {$0\mid 0$};
	\node[state] (f)[left of = e] {$0\mid 0$};
	
	\foreach \p/\q in {a/b, b/c, c/d, d/e, e/f}
	\draw[->] 
	(\q) edge (\p)
	(\p) edge (\q) 			
	;
	\end{tikzpicture}
	
	\begin{tikzpicture}[node distance=2cm]
	\node[state, fill=red, text=white, accepting] (a) {$1\mid 0$};
	\node[state] (b)[left of = a] {$0\mid 0$};
	\node[state] (c)[left of = b] {$1\mid 0$};
	\node[state] (d)[left of = c] {$0\mid 0$};
	\node[state] (e)[left of = d] {$0\mid 0$};
	\node[state] (f)[left of = e] {$0\mid 0$};
	
	\foreach \p/\q in {a/b, b/c, c/d, d/e, e/f}
	\draw[->] 
	(\q) edge (\p)
	(\p) edge (\q) 			
	;
	\end{tikzpicture}
	
\end{beispiel}

%-----------------------------------------------------------------------------

\end{document}
%=============================================================================
%%% Der Rest ist für meine Editor  (GNU Emacs, was sonst ;-) :
%%%
%%% Local Variables:
%%% TeX-command-default: "XPDFLaTeX"
%%% fill-column: 78
%%% TeX-master: t
%%% End:
