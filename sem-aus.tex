\documentclass[11pt]{article}

\usepackage{fontspec}
\usepackage[ngerman]{babel}
\usepackage[utf8]{inputenc}

\usepackage{amsmath}
% \usepackage{mathtools}

\usepackage[tt=false]{libertine}   % !!!!! das muss man nicht nutzen
\usepackage[libertine]{newtxmath}  % !!!!! das muss man nicht nutzen
\usepackage[supstfm=libertinesups,supscaled=1.2,raised=-.13em]{superiors} % params taken from doc

%
%\usepackage{tgpagella}
%\usepackage[euler-digits]{eulervm}
%
\usepackage{microtype}

\usepackage{fancyvrb}

%\usepackage{graphicx}

\usepackage{booktabs}
\usepackage[shortlabels]{enumitem}
\setlist{noitemsep}

\usepackage{titlesec}
%\usepackage{tcolorbox}
%\tcbuselibrary{listingsutf8}

\usepackage{bbold}
\newcommand{\Z}{\mathbb{Z}}


%-----------------------------------------------------------------------------
% für das Deckblatt

\usepackage{tikz}

\newcommand{\teilnehmername}{Vorname Nachname} % !!!!!
\newcommand{\teilnehmermatrnr}{1234567}        % !!!!!
\newcommand{\seminarart}{Proseminar}           % !!!!!  oder Seminar
\newcommand{\seminarlp}{3 LP}                  % !!!!!  Prosem immer 3 LP, Sem 3 oder 4 LP
\newcommand{\seminarjahr}{2017}                % !!!!!
%-----------------------------------------------------------------------------
\newcommand{\meta}[1]{$\langle$\textit{#1}$\rangle$}
\newcommand{\paket}[1]{\texttt{#1}}
\newcommand{\prgname}[1]{\texttt{#1}}

%-----------------------------------------------------------------------------
\author{Thomas Worsch}
\title{Hinweise zu Seminarausarbeitungen}

%=============================================================================
\begin{document}
%=======================================================================
% Anfang erste Seite
{\thispagestyle{empty}\large\sffamily\raggedright
%
\begin{tikzpicture}[remember picture,overlay]
  \coordinate[xshift=5mm,yshift=-5mm] (NW) at (current page.north west) {};
  \coordinate[xshift=-5mm,yshift=-5mm] (NE) at (current page.north east) {};
  \coordinate[xshift=-5mm,yshift=13mm] (SE) at (current page.south east) {};
  \coordinate[xshift=5mm,yshift=13mm] (SW) at (current page.south west) {};

  \draw[line width=0.25pt] (NW)
    [rounded corners=5mm] -- (NE) 
    [sharp corners] -- (SE)
    [rounded corners=5mm] -- (SW)
    [sharp corners] -- cycle
  ;
\end{tikzpicture}
%
\unskip % keine Ahnung warum das nötig ist
\noindent \textbf{\Large \seminarart\ (\seminarlp)} 
\\[\baselineskip]
%
Zellularautomaten und diskrete komplexe Systeme
% für Fortgeschrittene  % nur für das 4 Leistungspunkte Seminar !!!!!
\\[1ex]
%
im Sommersemester \seminarjahr

\vspace*{3\baselineskip}

\noindent \textbf{\Large Ausarbeitung} \\[\baselineskip]
%
von \textbf{\teilnehmername}, Matr.nr.~\teilnehmermatrnr

\vspace*{3\baselineskip}

\noindent \textbf{\Large Thema} \\[\baselineskip]
%
% nachfolgende ein Beispiel, für Konferenzbeiträge, Buchausschnitte, ...
% bitte analog vorgehen !!!!!
%
Karl Gustav von Kloppenstöck (1958)\\[1ex]
%
\textit{Zellularautomaten im ausgehenden Kambrium unter besonderer
  Berücksichtigung der Karlsruher Unterpflasterstraßenbahnen}\\[1ex]
%
Zentralblatt für historischen ÖPNV, Band \textbf{42}, S.~123-456
}
\clearpage
% Ende erste Seite
%=======================================================================
% Anfang zweite Seite
{\thispagestyle{empty}\raggedright

\noindent \textbf{\Large Erklärung}\\[1ex]
gemäß \S 6 (11) der Prüfungsordnung Informatik % !!!!! oder \S 6 (7) (bei MasterPO 2015)
(Bachelor) % oder Master !!!!!
\\[\baselineskip]

\noindent
Ich versichere wahrheitsgemäß, die Seminarausarbeitung zum
\seminarart{} "`Zellularautomaten und diskrete komplexe Systeme"' im
Sommersemester \seminarjahr{} selbstständig angefertigt, alle
benutzten Hilfsmittel vollständig und genau angegeben und alles
kenntlich gemacht zu haben, was aus Arbeiten anderer unverändert oder
mit Abänderungen entnommen wurde.

\vspace*{30mm}
\noindent
\begin{tabular}{@{}l}
  \hline
   \\[-1ex]
  \hbox to 0.6\textwidth{(\teilnehmername, Matr.nr.~\teilnehmermatrnr) \hss}
\end{tabular}
}
\clearpage
% Ende zweite Seite
%=======================================================================

%-----------------------------------------------------------------------------
\section{Einleitung}

Hoffentlich geht das gut: Das vorliegende Dokument soll gleichzeitig mehreren
Zwecken dienen:
%
\begin{itemize}[noitemsep]
\item Es soll als Vorlage für Seminarausarbeitungen dienen.
  %
  Daher stelle ich Ihnen die \LaTeX"=Quelle zur Verfügung.
  %
  Deshalb sind die beiden ersten Seiten auch etwas ungewöhnlich.
\item Es soll ein paar (meiner Meinung nach) wichtige Dinge zum Thema \LaTeX{}
  vermitteln.
\item Es soll dafür sorgen, dass der in den Seminarausarbeitungen benutzte
  Formalismus (halbwegs) einheitlich ist.
\end{itemize}
%
Daher ist die weitere Arbeit ist wie folgt aufgebaut: 

In Abschnitt~\ref{sec:grundlagen} sind noch mal die Grundlagen zu
Zellularautomaten aufgeschrieben. Die Auf"|forderung an Seminarteilnehmer ist:
%
\begin{itemize}[noitemsep]
\item Bitte benutzen Sie diesen Formalismus (soweit möglich und sinnvoll).
\item Aber bitte schreiben Sie ihn nicht alle noch mal ab, sondern verweisen
  Sie (erst mal) auf das vorliegende Dokument.
\item Allerdings ist auch mir noch unklar: Wie heißt dieses Dokument? Wie soll
  man das ziteren?
\end{itemize}
%
In Abschnitt~\ref{sec:dokument-struktur} wird erläutert, an welchen Stellen
man in diesem Dokument Dinge findet, die für den gewählten Aufbau von
(Pro-)Seminarausarbeitungen von Bedeutung sind.

In Abschnitt~\ref{sec:benutzte-pakete} findet man einige Anmerkungen zu den in
diesem \LaTeX"=Dokument benutzten Paketen. Da lernen Sie vielleicht noch das
ein oder andere für Ihr späteres Leben.

In Abschnitt~\ref{sec:no-go} werden einige böse Dinge aufgezählt, die man
\emph{niemals} tun soll.

In Abschnitt~\ref{sec:tipps} finden sich ein paar allgemeine Tipps zur
Abfassung von \LaTeX"=Quellen.

In der aktuellen Version dieses Dokumentes gibt es noch Lücken. Die, die ich
sehe, sind abschließend in Abschnitt~\ref{sec:todo} aufgeführt.

(Randbemerkung: Es ist eine gute und übliche Vorgehensweise, den ersten
einleitenden Abschnitt einer Arbeit wie eben geschehen mit einem Überblick
über den Rest zu beenden.)

%-----------------------------------------------------------------------------
\section{Grundlagen}
\label{sec:grundlagen}

$\Z$ bezeichnet die Menge der ganzen Zahlen. Sind $A$ und $B$ zwei Mengen, so
schreiben wir $B^A$ für die Menge aller Abbildungen der Form $f \colon A\to
B$.

Ein Zellularautomat ist festgelegt durch
%
\begin{enumerate}[noitemsep]
\item den zugrunde liegenden Raum $R$,
\item die endliche Zustandsmenge $Q$,
\item die endliche Nachbarschaft $N$ und
\item die lokale Überführungsfunktion
  \[
  \delta: Q^N \to Q
  \]
\end{enumerate}
%
Erläuterungen und abgeleitete Begriffe:
%
\begin{itemize}[noitemsep]
\item $R$ ist bei uns meist $\Z$ oder $\Z_m$ oder $\Z^2$
\item Formal enthält $N$ Koordinaten\emph{differenzen}.
\item Eine \emph{lokale Konfiguration} ist eine Abbildung $\ell:N\to Q$, also
  mit anderen Worten $\ell\in Q^N$.
\item Die lokale Überführungsfunktion induziert eine globale
  Überführungsfunktion
  \[
  \Delta: Q^R \to Q^R
  \]
\end{itemize}
%
Bitte malen Sie Raum"=Zeit"=Diagramme immer so, dass die Zeit \emph{von oben
  nach unten} zunimmt.


%-----------------------------------------------------------------------------

\end{document}
%=============================================================================
%%% Der Rest ist für meine Editor  (GNU Emacs, was sonst ;-) :
%%%
%%% Local Variables:
%%% TeX-command-default: "XPDFLaTeX"
%%% fill-column: 78
%%% TeX-master: t
%%% End:
