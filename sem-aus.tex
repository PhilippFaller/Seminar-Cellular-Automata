\documentclass[11pt]{article}

\usepackage{fontspec}
\usepackage[ngerman]{babel}
\usepackage[utf8]{inputenc}

\usepackage{amsmath}
% \usepackage{mathtools}
\usepackage[standard]{ntheorem}

\usepackage[tt=false]{libertine}   % !!!!! das muss man nicht nutzen
\usepackage[libertine]{newtxmath}  % !!!!! das muss man nicht nutzen
\usepackage[supstfm=libertinesups,supscaled=1.2,raised=-.13em]{superiors} % params taken from doc

%
%\usepackage{tgpagella}
%\usepackage[euler-digits]{eulervm}
%
\usepackage{microtype}

\usepackage{fancyvrb}

%\usepackage{graphicx}

\usepackage{booktabs}
\usepackage[shortlabels]{enumitem}
\setlist{noitemsep}

\usepackage{titlesec}
%\usepackage{tcolorbox}
%\tcbuselibrary{listingsutf8}

\usepackage{bbold}
\newcommand{\Z}{\mathbb{Z}}

\usepackage{csquotes}


%-----------------------------------------------------------------------------
% für das Deckblatt

\usepackage{tikz}
\usetikzlibrary{automata}

\newcommand{\teilnehmername}{Philipp Faller} % !!!!!
\newcommand{\teilnehmermatrnr}{1939704}        % !!!!!
\newcommand{\seminarart}{Proseminar}           % !!!!!  oder Seminar
\newcommand{\seminarlp}{3 LP}                  % !!!!!  Prosem immer 3 LP, Sem 3 oder 4 LP
\newcommand{\seminarjahr}{2017}                % !!!!!
%-----------------------------------------------------------------------------
\newcommand{\meta}[1]{$\langle$\textit{#1}$\rangle$}
\newcommand{\paket}[1]{\texttt{#1}}
\newcommand{\prgname}[1]{\texttt{#1}}
\newcommand{\defWord}[1]{\emph{#1}}

%-----------------------------------------------------------------------------
\author{Thomas Worsch}
\title{Hinweise zu Seminarausarbeitungen}

%=============================================================================
\begin{document}
%=======================================================================
% Anfang erste Seite
{\thispagestyle{empty}\large\sffamily\raggedright
%
\begin{tikzpicture}[remember picture,overlay]
  \coordinate[xshift=5mm,yshift=-5mm] (NW) at (current page.north west) {};
  \coordinate[xshift=-5mm,yshift=-5mm] (NE) at (current page.north east) {};
  \coordinate[xshift=-5mm,yshift=13mm] (SE) at (current page.south east) {};
  \coordinate[xshift=5mm,yshift=13mm] (SW) at (current page.south west) {};

  \draw[line width=0.25pt] (NW)
    [rounded corners=5mm] -- (NE) 
    [sharp corners] -- (SE)
    [rounded corners=5mm] -- (SW)
    [sharp corners] -- cycle
  ;
\end{tikzpicture}
%
\unskip % keine Ahnung warum das nötig ist
\noindent \textbf{\Large \seminarart\ (\seminarlp)} 
\\[\baselineskip]
%
Zellularautomaten und diskrete komplexe Systeme
% für Fortgeschrittene  % nur für das 4 Leistungspunkte Seminar !!!!!
\\[1ex]
%
im Sommersemester \seminarjahr

\vspace*{3\baselineskip}

\noindent \textbf{\Large Ausarbeitung} \\[\baselineskip]
%
von \textbf{\teilnehmername}, Matr.nr.~\teilnehmermatrnr

\vspace*{3\baselineskip}

\noindent \textbf{\Large Thema} \\[\baselineskip]
%
% nachfolgende ein Beispiel, für Konferenzbeiträge, Buchausschnitte, ...
% bitte analog vorgehen !!!!!
%
Angela Wu und Azriel Rosenfeld (1979)\\[1ex]
%
\textit{Cellular Graph Automata I }\\[1ex]
%
Information and Controll, Band \textbf{42}, S.~305-329
}
\clearpage
% Ende erste Seite
%=======================================================================
% Anfang zweite Seite
{\thispagestyle{empty}\raggedright

\noindent \textbf{\Large Erklärung}\\[1ex]
gemäß \S 6 (11) der Prüfungsordnung Informatik % !!!!! oder \S 6 (7) (bei MasterPO 2015)
(Bachelor) % oder Master !!!!!
\\[\baselineskip]

\noindent
Ich versichere wahrheitsgemäß, die Seminarausarbeitung zum
\seminarart{} "`Zellularautomaten und diskrete komplexe Systeme"' im
Sommersemester \seminarjahr{} selbstständig angefertigt, alle
benutzten Hilfsmittel vollständig und genau angegeben und alles
kenntlich gemacht zu haben, was aus Arbeiten anderer unverändert oder
mit Abänderungen entnommen wurde.

\vspace*{30mm}
\noindent
\begin{tabular}{@{}l}
  \hline
   \\[-1ex]
  \hbox to 0.6\textwidth{(\teilnehmername, Matr.nr.~\teilnehmermatrnr) \hss}
\end{tabular}
}
\clearpage
% Ende zweite Seite
%=======================================================================

%-----------------------------------------------------------------------------

\section{Übersicht}

\section{$d$"=Graphen}
\begin{definition}
Sei $L$ eine endliche, nicht leere Menge von Labels, mit einem ausgezeichneten Element $\#$. 
Ein \defWord{$d$-Graph} über $L$ ist ein 4"=Tupel $\Gamma = \left(N, A, f, g\right)$, wobei
\begin{itemize}
	\item[$N$] eine endliche, nicht leere Menge aus Knoten ist.
	\item[$A$] $\subseteq N \times N$ ist eine symmetrische Relation über $N$ und heißt \defWord{Kantenmenge} von $L$.
	\item[$f$] $: N \rightarrow L$ ist eine Abbildung, sodass für $\left(m, n\right) \in A$ gilt: $f\left(m\right) = \# \implies f\left(n\right) \neq \#$. $f$  heißt \defWord{Labelfunktion}.
	\item[$g$] $: A \rightarrow Z_d$ ist eine Abbildung, wobei $Z_d = \left \{1, 2,\text{\dots}, d \right \}$, die folgende Eigenschaften besitzt:
	
	Sei $A_n := \left \{\left(n, m\right) \in A\right \}$ die Menge aller Kanten, die von einem Knoten $n \in N$ ausgehen.
	\begin{enumerate}
		\item Falls $f\left(n\right) \neq \#$, ist $\left|A_n\right| = d$ und $g \vert_{A_n}$ ($g$ eingeschränkt auf $A_n$) ist eine Bijektion zwischen $A_n$ und $Z_d$.
		\item Falls $f\left(n\right) = \#$, ist $\left|A_n\right| = 1$
	\end{enumerate}
	
\end{itemize}
\end{definition}

\begin{beispiel}
	\begin{tikzpicture}[node distance=2cm]
	\node[state](a){$a$};
	\node[state](c)[below right of=a]{$c$};
	\node[state](b)[above right of=c]{$b$};
	\foreach \q/\p in {a/b, a/c, b/c}
			\draw[->] (\p) edge[bend left = 9] (\q)
					  (\q) edge[bend left = 9] (\p);
						
	\end{tikzpicture}
\end{beispiel}

\begin{beispiel}
	\begin{tikzpicture}[node distance=2cm]
		\node[state](a){$a$};
		\node[state](b)[right of = a]{$b$};
		\node[state](d)[right of = b]{$d$};
		\node[state](f)[below right of = d]{$f$};
		\node[state](e)[below left of = f]{$e$};
		\node[state](c)[left of = e]{$c$};
		\node[state](a1)[left of = a]{$\#$};
		\node[state](a2)[below left of = a]{$\#$};
		\node[state](c1)[left of = c]{$\#$};
		\node[state](f1)[above right of = f]{$\#$};
		\node[state](f2)[below right of = f]{$\#$};
		
		
		\foreach \p/\q in {a/b, b/d, b/c, d/e, d/f, e/f, c/e, a1/a, a2/a, c1/c, f1/f, f2/f}
			\draw[->] 
				(\q) edge[bend left=9] (\p)
				(\p) edge[bend left=9] (\q) 			
		;
	\end{tikzpicture}
\end{beispiel}

\section{Zellulare $d$"=Graph Automaten und Akzeptoren}

Ein \defWord{Zellulärer $d$"=Graph Automat} $M$ ist ein Tripel $\left(\Gamma, M, H\right)$, wobei
\begin{itemize}
	\item[$\Gamma$] ein $d$"=Graph $\left(N, A, f, g\right)$ über einer Labelmenge $L$ ist.
	\item[$M$]  ist ein endlicher Automat $\left(Q, \delta\right)$, mit 
	\begin{itemize}
		\item[$Q$] ist eine nicht leere Menge von Zuständen, mit $L \subseteq Q$.
		\item[$\delta$] $: Q \times Z_d^d \times Q^d \rightarrow \mathcal{P}\left(Q\right)$ ist eine Abbildung, für die gilt $\delta \left(\left \{\left(\#, z, q\right) \vert z \in Z_d^d, q \in Q^d\right \}\right) = \left \{\# \right \}$. $\delta$ heißt \defWord{Zustandsübergangsfunktion}.
	\end{itemize}
	\item[$H$] $: N \rightarrow Z_d^d$ ist eine Abbildung. Für $n \in N$ heißt das Bild $H \left(n\right) = \left(t_1, \text{\dots}, t_n\right) \in Z_d^d$ \defWord{Nachbarschaftsvektor} von $n$.
\end{itemize}

\section{Graphen Eigenschaften}

\section{Spannbaum durch Breitensuche}

\begin{beispiel}
	\begin{tikzpicture}[node distance=2cm]
		\node[state](a){$a\vert ?$};
		\node[state, fill=red, text=white, accepting](b)[right of = a]{$b$};
		\node[state](d)[right of = b]{$d\vert ?$};
		\node[state](f)[below right of = d]{$f\vert ?$};
		\node[state](e)[below left of = f]{$e\vert ?$};
		\node[state](c)[left of = e]{$c\vert ?$};
		\node[state](a1)[left of = a]{$\#$};
		\node[state](a2)[below left of = a]{$\#$};
		\node[state](c1)[left of = c]{$\#$};
		\node[state](f1)[above right of = f]{$\#$};
		\node[state](f2)[below right of = f]{$\#$};
		
		\foreach \p/\q in {a/b, b/d, b/c, d/e, d/f, e/f, c/e, a1/a, a2/a, c1/c, f1/f, f2/f}
			\draw[->] 
				(\q) edge[bend left=9] (\p)
				(\p) edge[bend left=9] (\q) 			
		;
		\end{tikzpicture}
		
		\begin{tikzpicture}[node distance=2cm]
		\node[state, fill=red, draw=none, text=white](a){$a\vert b$};
		\node[state, fill=blue, text=white, accepting](b)[right of = a]{$b$};
		\node[state, fill=red, draw=none, text=white](d)[right of = b]{$d\vert b$};
		\node[state](f)[below right of = d]{$f\vert ?$};
		\node[state](e)[below left of = f]{$e\vert ?$};
		\node[state, fill=red, draw=none, text=white](c)[left of = e]{$c\vert b$};
		\node[state](a1)[left of = a]{$\#$};
		\node[state](a2)[below left of = a]{$\#$};
		\node[state](c1)[left of = c]{$\#$};
		\node[state](f1)[above right of = f]{$\#$};
		\node[state](f2)[below right of = f]{$\#$};
		
		\foreach \p/\q in {a/b, b/d, b/c, d/e, d/f, e/f, c/e, a1/a, a2/a, c1/c, f1/f, f2/f}
			\draw[->] 
				(\q) edge[bend left=9] (\p)
				(\p) edge[bend left=9] (\q) 			
		;
		\end{tikzpicture}
		\newline
		\begin{tikzpicture}[node distance=2cm]
		\node[state, fill=yellow, draw=none](a){$a\vert b$};
		\node[state, fill=blue, text=white, accepting](b)[right of = a]{$b$};
		\node[state, fill=blue, draw=none, text=white](d)[right of = b]{$d\vert b$};
		\node[state, fill=red, draw=none, text=white](f)[below right of = d]{$f\vert d$};
		\node[state, fill=red, draw=none, text=white](e)[below left of = f]{$e\vert d$};
		\node[state, fill=blue, draw=none, text=white](c)[left of = e]{$c\vert b$};
		\node[state](a1)[left of = a]{$\#$};
		\node[state](a2)[below left of = a]{$\#$};
		\node[state](c1)[left of = c]{$\#$};
		\node[state](f1)[above right of = f]{$\#$};
		\node[state](f2)[below right of = f]{$\#$};
		
		\foreach \p/\q in {a/b, b/d, b/c, d/e, d/f, e/f, c/e, a1/a, a2/a, c1/c, f1/f, f2/f}
			\draw[->] 
				(\q) edge[bend left=9] (\p)
				(\p) edge[bend left=9] (\q) 			
		;
		\end{tikzpicture}
		\newline
		\begin{tikzpicture}[node distance=2cm]
		\node[state, fill=yellow, draw=none](a){$a\vert b$};
		\node[state, fill=blue, text=white, accepting](b)[right of = a]{$b$};
		\node[state, fill=yellow, draw=none](d)[right of = b]{$d\vert b$};
		\node[state, fill=yellow, draw=none](f)[below right of = d]{$f\vert d$};
		\node[state, fill=yellow, draw=none](e)[below left of = f]{$e\vert d$};
		\node[state, fill=yellow, draw=none](c)[left of = e]{$c\vert b$};
		\node[state](a1)[left of = a]{$\#$};
		\node[state](a2)[below left of = a]{$\#$};
		\node[state](c1)[left of = c]{$\#$};
		\node[state](f1)[above right of = f]{$\#$};
		\node[state](f2)[below right of = f]{$\#$};
		
		\foreach \p/\q in {a/b, b/d, b/c, d/e, d/f, e/f, c/e, a1/a, a2/a, c1/c, f1/f, f2/f}
		\draw[->] 
		(\q) edge[bend left=9] (\p)
		(\p) edge[bend left=9] (\q) 			
		;
		\end{tikzpicture}
		\newline
		\begin{tikzpicture}[node distance=2cm]
		\node[state, fill=yellow, draw=none](a){$a\vert b$};
		\node[state, fill=yellow, accepting](b)[right of = a]{$b$};
		\node[state, fill=yellow, draw=none](d)[right of = b]{$d\vert b$};
		\node[state, fill=yellow, draw=none](f)[below right of = d]{$f\vert d$};
		\node[state, fill=yellow, draw=none](e)[below left of = f]{$e\vert d$};
		\node[state, fill=yellow, draw=none](c)[left of = e]{$c\vert b$};
		\node[state](a1)[left of = a]{$\#$};
		\node[state](a2)[below left of = a]{$\#$};
		\node[state](c1)[left of = c]{$\#$};
		\node[state](f1)[above right of = f]{$\#$};
		\node[state](f2)[below right of = f]{$\#$};
		
		\foreach \p/\q in {a/b, b/d, b/c, d/e, d/f, e/f, c/e, a1/a, a2/a, c1/c, f1/f, f2/f}
		\draw[->] 
		(\q) edge[bend left=9] (\p)
		(\p) edge[bend left=9] (\q) 			
		;
		\end{tikzpicture}
		\newline
		\begin{tikzpicture}[node distance=2cm]
		\node[state, accepting](b){$b$};
		\node[state](c)[below of = b]{$c\vert b$};
		\node[state](d)[right of = c]{$d\vert b$};
		\node[state](a)[left of = c]{$a\vert b$};
		\node[state](a1)[below of = a]{$\#$};
		\node[state](a2)[left of = a1]{$\#$};
		\node[state](c1)[below of = c]{$\#$};
		\node[state](e)[right of = c1]{$e\vert d$};
		\node[state](f)[right of = e]{$f\vert d$};
		\node[state](f1)[below left of = f]{$\#$};
		\node[state](f2)[below right of = f]{$\#$};
		
		\foreach \p/\q in {a/b, b/d, b/c, d/e, d/f, a1/a, a2/a, c1/c, f1/f, f2/f}
		\draw[->] 
		(\q) edge[bend left=9] (\p)
		(\p) edge[bend left=9] (\q) 			
		;
		\end{tikzpicture}
		
\end{beispiel}

\section{Radius bestimmen und Label finden}

%-----------------------------------------------------------------------------

\end{document}
%=============================================================================
%%% Der Rest ist für meine Editor  (GNU Emacs, was sonst ;-) :
%%%
%%% Local Variables:
%%% TeX-command-default: "XPDFLaTeX"
%%% fill-column: 78
%%% TeX-master: t
%%% End:
