%% LaTeX-Beamer template for KIT design
%% by Erik Burger, Christian Hammer
%% title picture by Klaus Krogmann
%%
%% version 2.1
%%
%% mostly compatible to KIT corporate design v2.0
%% http://intranet.kit.edu/gestaltungsrichtlinien.php
%%
%% Problems, bugs and comments to
%% burger@kit.edu

\documentclass[18pt]{beamer}

%% SLIDE FORMAT

% use 'beamerthemekit' for standard 4:3 ratio
% for widescreen slides (16:9), use 'beamerthemekitwide'
\usepackage[utf8]{inputenc}
\usepackage[T1]{fontenc}
\usepackage{templates/beamerthemekit}
\usepackage{templates/tikzkit}
% \usepackage{templates/beamerthemekitwide}
\usepackage{mathtools}

%% TITLE PICTURE

% if a custom picture is to be used on the title page, copy it into the 'logos'
% directory, in the line below, replace 'mypicture' with the 
% filename (without extension) and uncomment the following line
% (picture proportions: 63 : 20 for standard, 169 : 40 for wide
% *.eps format if you use latex+dvips+ps2pdf, 
% *.jpg/*.png/*.pdf if you use pdflatex)

\titleimage{bauplan}

%% TITLE LOGO

% for a custom logo on the front page, copy your file into the 'logos'
% directory, insert the filename in the line below and uncomment it

\titlelogo{sdq_logo}

% (*.eps format if you use latex+dvips+ps2pdf,
% *.jpg/*.png/*.pdf if you use pdflatex)

%% TikZ INTEGRATION

% use these packages for PCM symbols and UML classes
% \usepackage{templates/tikzkit}
% \usepackage{templates/tikzuml}

% the presentation starts here

\title[Cellular Graph Automata I]{Cellular Graph Automata I}
\subtitle{Zellularautomaten und diskrete komplexe Systeme}
\author{Philipp Faller}
\institute{Worsch}

% Bibliography

\usepackage[citestyle=authoryear,bibstyle=numeric,hyperref,backend=biber]{biblatex}
\addbibresource{templates/example.bib}
\bibhang1em


\newcommand{\defWord}[1]{\emph{#1}}

\begin{document}

% change the following line to "ngerman" for German style date and logos
\selectlanguage{ngerman}

%title page
\begin{frame}
\titlepage
\end{frame}

%table of contents
% \begin{frame}{Gliederung}
% \tableofcontents
% \end{frame}

\section{$d$"~Graph}
\subsection{Definition}
\begin{frame}{$d$"~Graph}
	\begin{definition}[$d$"~Graph]
		Sei $L$ eine endliche, nicht leere Menge von Beschriftungen, mit ausgezeichneten Element $\#$. 
		Ein \defWord{$d$"~Graph} über $L$ ist ein 4"=Tupel $\Gamma = \left(N, A, f, g\right)$, wobei
		\begin{itemize}
			\item[$N$] endliche, nicht leere Menge aus Knoten 
			\item[$A$] $\subseteq N \times N$ eine symmetrische Relation über $N$ und heißt \defWord{Kantenmenge} von $L$.
			\item[$f$] $: N \rightarrow L$  heißt \defWord{Beschriftungsfunktion}.
			\item[$g$] $: A \rightarrow Z_d$ , mit $Z_d = \left \{1, 2,\text{\dots}, d \right \}$ 			
		\end{itemize}
	\end{definition}
\end{frame}

\begin{frame}
	\begin{definition}[$d$"~Graph]
		Weiter muss für $\left(n, m\right) \in A$ gelten:
		\begin{align*}
			f(n) = \# &\implies f(m) \neq \#
		\end{align*}
		
		Sei $A_n := \left \{\left(n, m\right) \in A\right \}$. Dann muss gelten:
		\begin{align*}
			f(n) \neq \# &\implies  \left|A_n\right| = d \text{ und } g \vert_{A_n} \text{ ist Bijektion} \\
			f(n) = \# &\implies\left|A_n\right| = 1
		\end{align*}
		
	\end{definition}
\end{frame}

\subsection{Beispiel}
\begin{frame}{Beispiel}
	\centering
		\begin{tikzpicture}[node distance=2cm, baseline=(current bounding box.north), auto]
		\node[state](a){$a$};
		\node[state](c)[below right of=a]{$c$};
		\node[state](b)[above right of=c]{$b$};
		\foreach \q/\p/\a/\b in {a/b/1/1, a/c/2/1, b/c/2/2}
		\draw[->] (\q) edge[bend left=9] node {\a} (\p) 
		(\p) edge[bend left=9] node {\b} (\q) 
		;
		
		\end{tikzpicture}
\end{frame}


\begin{frame}{Beispiel}
	\centering
		\begin{tikzpicture}[node distance=2cm, baseline=(current bounding box.north), auto]
		\node[state](a){$a$};
		\node[state](b)[right of = a]{$b$};
		\node[state](d)[right of = b]{$d$};
		\node[state](f)[below right of = d]{$f$};
		\node[state](e)[below left of = f]{$e$};
		\node[state](c)[left of = e]{$c$};
		\node[state](a1)[left of = a]{$\#$};
		\node[state](c1)[left of = c]{$\#$};
		\node[state](a2)[left of = c1]{$\#$};
		\node[state](f1)[right of = f]{$\#$};
		
		
		\foreach \p/\q/\a/\b in {a/b/1/2, b/d/1/2, b/c/2/3, d/e/2/3, d/f/1/2, e/f/3/3, c/e/1/3, a1/a/1/1, a2/a/3/1, c1/c/1/1, f1/f/2/1}
		\draw[->] 
		(\q) edge[bend left=9] node {\a} (\p) 
		(\p) edge[bend left=9] node {\b} (\q) 			
		;
		\end{tikzpicture}
\end{frame}

\subsection{Weitere Definitionen}
\begin{frame}{Zugrunde liegender Graph}
	\begin{definition}[Zugrunde liegender Graph]
		Sei $\Gamma$ ein $d$"~Graph. 
		Der Graph $G(\Gamma) = \left(V, E\right)$ mit
		\begin{align*}
			V &= N\setminus \left\{n \in N \mid f\left(n\right) = \# \right\} \\		
			E &= A\setminus \left\{\left(n, m\right) \in A \mid f(n) = \# \lor f(m) = \# \right\}
		\end{align*} 
		heißt \defWord{$\Gamma$ zugrunde liegende Graph}.
		
		
		Ein $d$"~Graph heißt \defWord{zusammenhängend}, wenn der zugrunde liegende Graph zusammenhängend ist.
	\end{definition}
\end{frame}

\begin{frame}{Nachbarschaft}
	\begin{definition}[Nachbarschaft]
		Sei $\left(n, m\right) \in A$ und $g\left(n, m\right) = i$. 
		Dann heißt $m$ der \defWord{$i$"=te Nachbar von $n$}.
		Außerdem gilt: 
		\begin{displaymath}
		\left(m, n\right) \in A \text{ und }\exists j \in Z_d : g(m, n) = j.
		\end{displaymath}
		Dann ist $n$ der $j$"=te Nachbar von $m$. 
		Definiere $h : N \times Z_d \rightarrow N$, so dass
		\begin{displaymath}
		h(n, i) = m \iff g(n, m) = i.
		\end{displaymath}
	\end{definition}
\end{frame}

\section{Zellulare $d$"~Graph Automaten}
\begin{frame}{Zellulare $d$"~Graph Automaten}
	\begin{definition}[$d$"~Graph Automat]
		Ein \defWord{Zellulärer $d$"~Graph Automat} $\mathcal{M}$ ist ein Tripel $\left(\Gamma, M, H\right)$, wobei
		\begin{itemize}
			\item[$\Gamma$] ein $d$"~Graph $\left(N, A, f, g\right)$ über Menge von Beschriftungen $L$
			\item[$M$]  ist endlicher Automat $\left(Q, \delta\right)$, mit 
			\begin{itemize}
				\item[$Q$] ist endliche, nicht leere Menge von Zuständen, mit $L \subseteq Q$.
				\item[$\delta$] $: Q \times Z_d^d \times Q^d \rightarrow \mathcal{P}\left(Q\right)$  heißt \defWord{Zustandsübergangsfunktion}.
			\end{itemize}
			\item[$H$] $: N \rightarrow Z_d^d$. 
			Für $n \in N$ heißt $H(n) = \left(t_1, \text{\dots}, t_n\right) \in Z_d^d$ \defWord{Nachbarschaftsvektor} von $n$. 
		\end{itemize}
	\end{definition}
\end{frame}
%TODO Dieser Frame weg, erstes Ding auf Folie davor. Zweites durch BSP ersetzten.
\begin{frame}{Zellulare $d$"~Graph Automaten}
	\begin{definition}[$d$"~Graph Automat]
	Für $\delta$ muss weiter gelten: 
	\begin{displaymath}
		\delta \left(\#, Z_d^d, Q^d\right) = \left \{\# \right \}
	\end{displaymath}
	Der Nachbarschaftsvektor hat die Gestallt:
	\begin{displaymath}
	\forall i \in Z_d : H(n)_i = 
	\begin{dcases}
	g(h(n, i), n) & ,f(n) \neq \# \\
	1 & , f(n) = \# \\
	\end{dcases}
	\end{displaymath}
	\end{definition}
\end{frame}
\section{Spannbaum durch Breitensuche}
\begin{frame}{Spannbaum durch Breitensuche}
	
\end{frame}
\section{Radius bestimmen}
\begin{frame}{Radius bestimmen}
	
\end{frame}

\section{Beschriftung finden}
\begin{frame}{Beschriftung finden}
	
\end{frame}
\end{document}
